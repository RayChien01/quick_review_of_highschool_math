\documentclass[12pt,a4paper]{article}

\usepackage{graphics}
\usepackage[margin = 3 cm]{geometry}
\usepackage{mathtools,amssymb}
\usepackage{float}
\usepackage{tikz}
\usepackage{tikz-3dplot}
\usetikzlibrary{calc, decorations.pathreplacing}

% Equation numbering follows section
\numberwithin{equation}{section}

\usepackage{caption}
\captionsetup{skip=10pt} % Set spacing between caption and content below (tables, figures)

\usepackage{titlesec}
\titleformat{\section}{\normalfont\Large\bfseries}{\thesection}{1em}{}


\usepackage[colorlinks = true,
    linkcolor = black,    % TOC and section title links = black
    citecolor = blue,     % Citation links = blue
    urlcolor = blue,      % URL links = blue
    filecolor = blue,     % File links = blue
    anchorcolor = blue    % Other anchors = blue
]{hyperref}

\usepackage[
    type={CC},
    modifier={by-nc-sa},
    version={4.0},
    hyphenation={raggedright}
]{doclicense}



\begin{document}

    \title{Quick Review of High School Mathematics}
    \author{Ray Chien}
    \maketitle

    \tableofcontents

    \doclicenseThis

    \section{Book 1}

    \subsection{Distance of a Point From a Line}

    \[\frac{\left| a x_0 + b y_0 + c\right|}{\sqrt{a^2 + b^2}}\]

    \subsection{Distance Between Two Parallel Lines}

    \[\frac{\left| c_1 - c_2 \right|}{\sqrt{a^2 + b^2}}\]

    \subsection{Equation of a Circle Given The Endpoints of a Diameter}

    \[(x - x_1)(x - x_2) + (y - y_1)(y - y_2) = 0 \]

    \subsection{Equations of a Tangent Line or a Chord of Contact on A Circle}

    \[(x - h)(x_0 - h) + (y - k)(y_0 - k) = r^2\]

    \[x_0 x + y_0 y + d \frac{x + x_0}{2} + e \frac{y + y_0}{2} + f = 0\]

    \subsection{Lagrange Interpolation}

    \[\sum_{j=0}^{n} f(x_j) \left( \prod_{\substack{i=0 \\ i \neq j}}^{n} \frac{x - x_i}{x_j - x_i} \right)\]


    \section{Book 2}

    \subsection{Series}

        \[\sum_{k=1}^{n} ar^{k-1} = \frac{a(1-r^n)}{1-r}\]

        \[\sum_{k=1}^{n} k = \frac{n(n+1)}{2}\]

        \[\sum_{k=1}^{n} k^2 = \frac{n(n+1)(2n+1)}{6}\]

        \[\sum_{k=1}^{n} k^3 = \left[ \frac{n(n+1)}{2} \right]^2\]

    \subsection{Percentiles}

        \begin{tikzpicture}[scale=1.5]
            % 1. Draw the main horizontal line
            \draw[thick, -] (0,0) -- (10,0);

            % 2. Loop to draw the "Outside" numbers (0 to 5) and tick marks
            % These represent the Data Points or Indices (Rank)
            \foreach \x in {0,1,2,3,4,5} {
                \draw[thick] (2*\x, 0.15) -- (2*\x, -0.15); % Draw tick
                \node[below] at (2*\x, -0.25) {\textbf{\x}}; % Label 0,1,2...
            }
            
            % Label for the bottom part
            \node[below=0.9cm] at (5,0) {\small Data Indices / Ranks ($i$)};

            % 3. Loop to draw the "Inside" numbers (1 to 5)
            % These represent the Intervals or Percentile Sections
            \foreach \x [count=\i] in {0,1,2,3,4} {
                % Calculate position: halfway between x and x+1
             
                \node[above, blue] at (2*\x + 1, 0.1) {\textbf{\i}}; 
            }

            % Label for the top part
            \node[above=0.6cm, blue] at (5,0) {\small Percentile Intervals / Regions};
        \end{tikzpicture}

        \subsection{Variance and Standard Deviation}

        \[\sigma^2 = \frac{1}{n}\sum_{i = 1}^{n} (x_i - \mu)^2 = \frac{1}{n} \sum_{i = 1}^{n} x_i^2 - \mu^2\]

        \[\sigma = \sqrt{\frac{1}{n}\sum_{i = 1}^{n} (x_i - \mu)^2} = \sqrt{\frac{1}{n} \sum_{i = 1}^{n} x_i^2 - \mu^2}\]

        \subsection{Correlation Coefficient}

        \[r = \frac{1}{n}\sum_{i = 1}^{n} x_i'y_i' = \frac{S_{xy}}{\sqrt{S_{xx}}\sqrt{S_{yy}}}\]

        Note: The numerator is proportional to $n$, and the denominator is proportional to $\sqrt{n} \times \sqrt{n} = n$, so no extra multiplication or division by $n$ is required.

        \subsection{Line of Best Fit}

        \[y - \mu_y = r \times \frac{\sigma_y}{\sigma_x} (x - \mu_x) = \frac{S_{xy}}{S_{xx}} (x - \mu_x)\]

        \subsection{Pascal's Principle}

        \[C^n_k = C^{n-1}_{k-1} + C^{n-1}_k\]

        \subsection{Combination With Repetition}

        \[H^n_k = C^{n+k-1}_{k}\]

        \subsection{Derangement}

        \[D_n = n! \sum_{k = 0}^{n} \frac{(-1)^k}{k!}\]

        \subsection{Sine Rule}

        \[\frac{a}{\sin{A}} = \frac{b}{\sin{B}} = \frac{c}{\sin{C}} = 2R\]

        \subsection{Cosine Rule}

        \[a^2 = b^2 + c^2 -2bc\cos A\]

    \section{Book 3}

        \subsection{Projection Vector And Its Length}
        
            Projection Vector:
            \[\frac{\vec{a} \cdot \vec{b}}{|\vec{b}|} \cdot \frac{\vec{b}}{|\vec{b}|}\]
            
            \noindent Length of Projection Vector:
            \[\frac{|\vec{a} \cdot \vec{b}|}{|\vec{b}|}\]

            \subsection{Triangle Inequality}

            \[\left| |x| - |y| \right| \le |x \pm y| \le |x| + |y|\]

            \subsection{Cauchy-Schwarz Inequality}

            \[(a_1^2 + a_2^2)(b_1^2 + b_2^2) \geq (a_1b_1 + a_2b_2)^2\]

            The equality holds if and only if $a_1b_2 = a_2b_1$

        \subsection{Two-Dimensional Determinant}

            \[% (2) 對調變號
            \begin{vmatrix} a & b \\ c & d \end{vmatrix} = - \begin{vmatrix} b & a \\ d & c \end{vmatrix}; \quad
            \begin{vmatrix} a & b \\ c & d \end{vmatrix} = - \begin{vmatrix} c & d \\ a & b \end{vmatrix}
            \]
            \[% (3) 成比例為 0
            \begin{vmatrix} ra & a \\ rc & c \end{vmatrix} = 0; \quad
            \begin{vmatrix} ra & rb \\ a & b \end{vmatrix} = 0
            \]
            \[% (4) 提出倍數
            \begin{vmatrix} ra & b \\ rc & d \end{vmatrix} = r \begin{vmatrix} a & b \\ c & d \end{vmatrix}; \quad
            \begin{vmatrix} ra & rb \\ c & d \end{vmatrix} = r \begin{vmatrix} a & b \\ c & d \end{vmatrix}
            \]
            \[% (5) 乘 k 倍加到另一行/列,值不變
            \begin{vmatrix} a & ka+b \\ c & kc+d \end{vmatrix} = \begin{vmatrix} a & b \\ c & d \end{vmatrix}; \quad
            \begin{vmatrix} a & b \\ ka+c & kb+d \end{vmatrix} = \begin{vmatrix} a & b \\ c & d \end{vmatrix}
            \]
            \[% (6) 拆開相加
            \begin{vmatrix} a+x & b \\ c+y & d \end{vmatrix} = \begin{vmatrix} a & b \\ c & d \end{vmatrix} + \begin{vmatrix} x & b \\ y & d \end{vmatrix}; \quad
            \begin{vmatrix} a+x & b+y \\ c & d \end{vmatrix} = \begin{vmatrix} a & b \\ c & d \end{vmatrix} + \begin{vmatrix} x & y \\ c & d \end{vmatrix}
            \]

        \subsection{Cramer's Rule ($2 \times 2$)}

            \[
            \begin{cases}
            a_1 x + b_1 y = c_1 \\
            a_2 x + b_2 y = c_2
            \end{cases}
            \]


            \[
            \Delta = \begin{vmatrix} a_1 & b_1 \\ a_2 & b_2 \end{vmatrix}, \quad
            \Delta_x = \begin{vmatrix} c_1 & b_1 \\ c_2 & b_2 \end{vmatrix}, \quad
            \Delta_y = \begin{vmatrix} a_1 & c_1 \\ a_2 & c_2 \end{vmatrix}
            \]


            \begin{table}[h]
                \centering
                \begin{tabular}{|c|c|l|}
                    \hline
                    \textbf{Condition} & \textbf{Solution} & \textbf{Geometric Meaning} \\
                    \hline
                    $\Delta \neq 0$ & Unique ($x=\frac{\Delta_x}{\Delta}, y=\frac{\Delta_y}{\Delta}$) & Intersecting lines \\
                    \hline
                    $\Delta = 0, \, (\Delta_x \neq 0 \lor \Delta_y \neq 0)$ & None & Parallel distinct lines \\
                    \hline
                    $\Delta = \Delta_x = \Delta_y = 0$ & Infinite & Coincident lines \\
                    \hline
                \end{tabular}
            \end{table}
        
    \section{Book 4}

        \subsection{Theorem of Three Perpendiculars}

            Let line $AB$ be perpendicular to plane $E$ at point $B.$ If on plane $E$ the line $BC$ is perpendicular to line $L$ at $C$, then line $AC$ is perpendicular to line $L$ at point $C.$

            \tdplotsetmaincoords{60}{120}
            \begin{tikzpicture}[tdplot_main_coords, scale=1.5]
                % Define coordinates
                \coordinate (B) at (0,0,0);
                \coordinate (A) at (0,0,3);
                \coordinate (C) at (2,0,0);
                \coordinate (L_start) at (2,-2,0);
                \coordinate (L_end) at (2,3,0);

                % Draw Plane alpha
                \filldraw[fill=gray!10, draw=gray] (-1,-2,0) -- (4,-2,0) -- (4,4,0) -- (-1,4,0) -- cycle;
                \node at (3.5, 3.5, 0) {E};

                % Draw Line L
                \draw[thick, red] (L_start) -- (L_end) node[right] {$L$};

                % Draw Connections
                \draw[dashed] (B) -- (A); % AB (Vertical)
                \draw[thick] (B) -- (C); % BC (Projection)
                \draw[thick] (A) -- (C); % AC (Slant)

                % Right Angle Markers
                % At B (AB perpendicular to plane)
                \draw (0,0,0.3) -- (0.3,0,0.3) -- (0.3,0,0);
                
                % At C (BC perpendicular to L, assuming coordinates align)
                \draw (1.7,0,0) -- (1.7,0.3,0) -- (2,0.3,0);

                % Points (Draw last to be on top)
                \fill (B) circle (1.5pt) node[below, yshift=-5pt] {$B$};
                \fill (A) circle (1.5pt) node[above] {$A$};
                \fill (C) circle (1.5pt) node[below right] {$C$};

            \end{tikzpicture}

        \subsection{Cross Product}

        \[(a_1, a_2, a_3) \times (b_1, b_2, b_3) = \left( \begin{vmatrix} a_2 & a_3 \\ b_2 & b_3 \end{vmatrix}, \begin{vmatrix} a_3 & a_1 \\ b_3 & b_1 \end{vmatrix}, \begin{vmatrix} a_1 & a_2 \\ b_1 & b_2 \end{vmatrix} \right)\]

        \subsection{Volume of Parallelepipeds}

            \[|\vec{a} \cdot \vec{b} \times \vec{c}| = |
            \begin{vmatrix}
                a_1 & a_2 & a_3\\
                b_1 & b_2 & b_3\\
                c_1 & c_2 & c_3
            \end{vmatrix}|\]

            \subsection{Laplace Expansion}

            \[
            \begin{vmatrix}
                + & - & +\\
                - & + & -\\
                + & - & +
            \end{vmatrix}\]

        \subsection{Point-Normal Form}

            \[a(x - x_0) + b(y - y_0) + c(z - z_0) = 0\]
        
        \subsection{Distance of a Point From a Plane}

            \[\frac{|ax_0 + by_0 + cz_0 + d|}{\sqrt{a^2 + b^2 + c^2}}\]

        \subsection{Distance Between Two Parallel Planes}

            \[\frac{|d_1 - d_2|}{\sqrt{a^2 + b^2 + c^2}}\]

        \subsection{Lines in Space}

            \[
            \begin{cases}
            x = x_0 + at\\
            y = y_0 + bt\\
            z = z_0 + ct\\
            \end{cases},t \in \mathbb{R}
            \]

            \[\frac{x - x_0}{a} = \frac{y - y_0}{b} = \frac{z - z_0}{c}\]

        \subsection{Conditional Probability}

        \[P(B|A) = \frac{n(A \cap B)}{n(A)} = \frac{P(A \cap B)}{P(A)}\]

        \[P(A \cap B) = P(B)P(A | B) = P(A)P(B | A)\]

        \[P(B) = P(A' \cap B) + P(A \cap B)\]

        \subsection{Independent Events}

        \[P(A \cap B) = P(A)P(B)\]

        If the events $A, B$ are independent, then $A', B$ are independent; $A, B'$ are independent; and $A', B'$ are independent.

        \subsection{Bayes' Theorem}

        \[P(A | B) = \frac{P(A)P(B | A)}{P(B)} = \frac{P(A) P(B | A)}{P(A)P(B | A) + P(A')P(B | A')}\]

        \subsection{Operational Properties of Matrices}

        \[(AB)C = A(BC)\]

        \[A(B + C) = AB + AC\]

        \[(A + B)C = AC + BC\]

        \subsection{Inverse Matrix}

        \[A^{-1} = \begin{bmatrix}
            a & b\\ c & d
        \end{bmatrix}^{-1} = \frac{1}{\det(A)}
        \begin{bmatrix}
            d & -b\\ -c & a
        \end{bmatrix}\]

        \subsection{Scaling Matrix}

        \[\begin{bmatrix}
            h & 0\\ 0 & k
        \end{bmatrix}\]

        \subsection{Shear Matrix}

        \[\begin{bmatrix}
            1 & h\\ 0 & 1
        \end{bmatrix} \quad \begin{bmatrix}
            1 & 0\\ k & 1
        \end{bmatrix}\]

        \subsection{Rotation Matrix}
        
        \[\begin{bmatrix}
            \cos \theta & - \sin \theta\\ \sin \theta & \cos \theta
        \end{bmatrix}\]
        
        \subsection{Reflection Matrix}

        \[\begin{bmatrix}
            \cos 2\theta &  \sin 2\theta\\ \sin 2\theta & -\cos 2\theta
        \end{bmatrix}\]

        \subsection{Transition Matrix}

        \[\begin{bmatrix}
            a &  b\\  1 - a & 1 - b
        \end{bmatrix}\]

        where $0 \leq a, b \leq 1$

    \section{Additional Tricks}

        \subsection{Centroids of Triangles}

            \[\overrightarrow{DG} = \frac{1}{3}\overrightarrow{DA} + \frac{1}{3}\overrightarrow{DB} + \frac{1}{3}\overrightarrow{DC}\]

        \subsection{Circumcenters of Triangles}

            \[\overrightarrow{AB} \cdot \overrightarrow{AO} = \frac{1}{2}\overline{AB}^2\]

        \subsection{Incenters of Triangles}

            \[\overrightarrow{DI} = \frac{\overline{BC}}{\overline{AB} + \overline{BC} + \overline{CA}} \overrightarrow{DA} + \frac{\overline{CA}}{\overline{AB} + \overline{BC} + \overline{CA}} \overrightarrow{DB} + \frac{\overline{AB}}{\overline{AB} + \overline{BC} + \overline{CA}} \overrightarrow{DC}\]

        \subsection{Orthocenters of Triangles}

            \[\overrightarrow{AB} \cdot \overrightarrow{AC} = \overrightarrow{AB} \cdot \overrightarrow{AH} = \overrightarrow{AC} \cdot \overrightarrow{AH}\]

        \subsection{The Pairing Method}

            While dividing some labeled items into groups of two, pick the partners for the unpaired items. For example, the numbers of ways to pair 5 people and one person left is $C^5_1 \times 3 \times 1 = 15.$

        \subsection{Stewart's Theorem}

            \[z^2= \frac{n}{m + n} \, x^2 + \frac{m}{m + n} \, y^2 -mn\]

            \begin{center}
            \begin{tikzpicture}[scale=1.5, thick]
                % 1. Define Coordinates
                % You can adjust these points to change the shape of the triangle
                \coordinate (A) at (2.5, 4); % Top vertex
                \coordinate (B) at (0, 0);   % Bottom left vertex
                \coordinate (C) at (7, 0);   % Bottom right vertex
                % Point D lies on BC. Let's place it at x=3.
                \coordinate (D) at (3, 0);

                % 2. Draw the main triangle and the cevian line
                \draw (A) -- (B) -- (C) -- cycle; % Main triangle ABC
                \draw (A) -- (D);           % Cevian AD

                % 4. Label lengths x, y, z
                % 'sloped' makes the text align with the line segment
                \path (A) -- (B) node[midway, above left] {$x$};
                \path (A) -- (C) node[midway, above right] {$y$};
                \path (A) -- (D) node[midway, right] {$z$};

                % 5. Label lengths m, n
                \path (B) -- (D) node[midway, below] {$m$};
                \path (D) -- (C) node[midway, below] {$n$};

            \end{tikzpicture}
            \end{center}

            When the dividing line is the bisector, $z^2 = xy - mn$.

            When it is the median, $2(z^2+n^2) = x^2 + y^2.$

        \subsection{Vectors and Areas about a Point inside a Triangle}

            \[S_A \vec{PA} + S_B \vec{PB} + S_C \vec{PC} = \vec{0}\]

            \begin{center}
            \begin{tikzpicture}[scale=1.2]
                % Define coordinates
                \coordinate (A) at (1, 4);
                \coordinate (B) at (0, 0);
                \coordinate (C) at (5, 0);
                \coordinate (P) at (2, 1.5); % Arbitrary point inside

                % Draw Triangle ABC
                \draw[thick] (A) -- (B) -- (C) -- cycle;

                % Draw internal vectors/lines
                \draw[thick, ->, blue] (P) -- (A) node[midway, left] {$\vec{PA}$};
                \draw[thick, ->, blue] (P) -- (B) node[midway, above left] {$\vec{PB}$};
                \draw[thick, ->, blue] (P) -- (C) node[midway, above right] {$\vec{PC}$};
                
                % Label Vertices
                \node[above] at (A) {$A$};
                \node[below left] at (B) {$B$};
                \node[below right] at (C) {$C$};
                \node[above right] at (P) {$P$};

                % Label Areas (approximate positions)
                \node at ($(P)!0.5!(B)!0.5!(C)$) {\large $S_A$};
                \node at ($(P)!0.5!(A)!0.5!(C)$) {\large $S_B$};
                \node at ($(P)!0.5!(A)!0.5!(B)$) {\large $S_C$};

            \end{tikzpicture}
            \end{center}

        \subsection{Areas of Triangles in Terms of Circumradius}

        \[Area = \frac{abc}{4R}\]

    \subsection{Shoelace Formula}

        \[Area = \frac{1}{2}\,|\begin{vmatrix}
            x_1 & x_2 & \ldots & x_k & x_1\\
            y_1 & y_2 & \ldots & y_k & y_1
        \end{vmatrix}|\]

    \subsection{Brahmagupta–Fibonacci Identity}

    \[(a^2 + b^2)(c^2 + d^2) = (ac + bd)^2 + (ac - bd)^2\]

    The identity can be a trivial proof of the two-dimensional Cauchy–Schwarz inequality.

\end{document}